In our ACO and GA algorithms we defined how the robot should fire and move its gun but never when to fire its gun. When training the combination of strategies we used a default strategy which was quite effective.

Whenever a robot was scanned, we saved its (X, Y) coordinates. In the main loop when deciding whether to fire or not, we calculated the angle between the enemy's most recently scanned location and robot's current location. The difference $\Delta$ between this angle and our robot's current gun heading is used to decide whether or not to shoot. The default policy is to shoot iff $-10 \leq \Delta \leq 10$, and if the last scan was less than 5 turns old.

After learning a good strategy using the ACO or GA methods we applied Neuroevolution to learn the fire function. The inputs were $\Delta$, and the age in turns of the last scan. The output was a boolean: fire or don't fire. The same implementation of Neuroevolution described in the previous section was used with the network topology detailed in Figure \ref{neurotuning}.

\FigNeurotuning
