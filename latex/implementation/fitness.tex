\subsubsection{Strategy Learning}

Fitness calculations were performed for each candidate \texttt{StrategySolution} by writing the selected strategies to file. A generalised robot class \texttt{ScriptBot} was created which reads the strategies to file, creates instances of the selected strategies using a \texttt{StrategyFactory} class, and executes them when required by the Robocode environment.

\newcommand{\BattleRunner}{\texttt{BattleRunner}\,}

Each battle is handled by an instance of an implementation of the Robocode \\ \texttt{BattleAdaptor} called \BattleRunner. Only one instance of \BattleRunner  is used throughout all the fitness calculations for every generation. Each instance creates an instance of the \texttt{RobocodeEngine}. Initializing the \texttt{RobocodeEngine} takes time; and therefore by only constructing one instance time is saved.

Fitnesses were calculated sparingly. Elite members carried to the next generations did not have their fitnesses calculated. Further, in the GA, mutant genotypes and crossed-over offspring only have their fitnesses calculated if they are not the clone of a member already in the next population.

In order to tune the best found strategy, a different class \texttt{NNScriptBot} was used which not only loads the strategies akin to \texttt{ScriptBot}; but also loads a neural network specification which describes the topology of the neural network and the weights of each of the corresponding edges.

The \BattleRunner class when executing a battle passes in a string representing the class name of the enemy, for example ``\texttt{sample.Tracker}'' and the string representing the corresponding script bot, ``\texttt{sample.ScriptBot}''.

\subsubsection{Robot Training using Neuroevolution}

The fitnesses of robots trained using Neuroevolution were computed in a similar manner to \texttt{NNScriptBot}; using a class called \texttt{NNBot}. A specification was loaded and executed. 

\subsubsection{Issues Faced using the Robocode API}
Unfortunately the Robocode API cannot accept instances of \texttt{Robot}s in a ``\texttt{runBattle}'' function. The File input and output described above was used as a work around to specify learned behaviours. This disk I/O turned out to not be such a bottleneck for learning; but it did make the fitness calculations more awkward to compute.  As such they were abstracted by a \texttt{GeneralFitness} interface which takes in a candidate and outputs the score ratio.
