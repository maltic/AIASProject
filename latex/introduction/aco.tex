ACO is a metaheuristic inspired on the foraging behaviour of Argentine ants. Our ACO is based loosely on the Max-Min Ant System in \cite{stutzle2000max}. We keep a finite population of ants which walk through a graph representing the solution space of the problem. The way these ants choose nodes is the first important feature that makes ACO an effective algorithm. Initially, an ant will compute how much interest it has in a node. This is described in Equation \ref{eqn:aco:initialfitness}.

\begin{equation}
\label{eqn:aco:initialfitness}
\text{Interest} = r \times m + p
\end{equation}

\noindent where the variable $r$ is sampled uniformly from $[0, 1)$. The variable $m$ is a randomness factor chosen by the programmer. In our implementation this was set to 1.0 by default. Finally, $p$ is defined as the amount of pheromone on a node. The Max-Min Ant system constrains the value pf $p$ to $[0, 1]$. 

The ant then walks to the node with the maximum interest. It continues to do this until it has reached maximum depth. After all ants have walked through the graph, the amount of pheromone on each node of the graph decreases by a small amount (chosen to be 0.05 in our implementation).

The fitness is calculated for all of the ``ant walks'' and the fittest ant is allowed to lay down some new pheromone. It does this by putting down $Q/L$ pheromone on each node where $Q$ is the total amount of pheromone an ant carries (chosen to be 1.0) and $L$ is the number of nodes in the solution. Our `solution graph' was very simple and is inspired by the work in \cite{emergentbehaviours}. A solution was represented by a vector of four numbers $(a, b, c, d)$. Each of these numbers is an identifier for a pre-defined strategy. Each identifier represents a strategy for movement, firing, scanning/radar and a gun movement strategy respectively.


A summary of the strategies we implemented can be found in Table \ref{table:strategies}.

\begin{table*}
\centering
\begin{tabular}{|l|l|}
\hline
Movement & TODO \\
\hline
Firing & TODO \\
\hline
Scanning & TODO \\
\hline
Gun Movement & TODO \\
\hline
\end{tabular}
\caption{A Summary of the Strategies Combined in to form the Solution-Space of the Ant Colony Optimization and Genetic Algorithms}
\label{table:strategies}
\end{table*}

An ant walks through the solution graph by selecting a value for $a, b, c,$ and then $d$. Clearly this is the same process as that used by Hong and Cho in \cite{emergentbehaviours}.

We have made some modifications to the method. Firstly, we allowed a strategy to access all the data with which the robot has access. We also removed the target selection strategy as we are only focussing on one versus one battles. We also removed the `avoid' strategies.


